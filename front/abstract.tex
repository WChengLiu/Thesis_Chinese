% !TeX root = ../main.tex

\begin{abstract*}
自由電子雷射是一種以高品質電子束作為介質,讓接近光速的電子束在周期性磁場中受到激發且放大電磁輻射的新型雷射光源,
這需要高品質和高穩定的電子束,雷射電漿交互作用中電子的加速已經進行了二十多年的實驗研究,並被廣泛認為是能夠替代傳統射頻加速器的方案。目前在厘米等級的雷射電漿波電子加速器,已經被證實可以將電子加速至十億電子伏特,並具備發散角小穩定性高的電子脈衝,有很大的潛力投入自由電子雷射應用。
本論文所呈現的是透過二維粒子式模擬(Particle-In-Cell simulation)來研究電漿尾流場加速衝擊波注入電子的物理特性。當前電漿源設計方面努力針對兩個主題。一個主題是開發控制捕捉電漿尾流電子的方法。另一個是開發電漿波導以最大化加速長度。在結合可控注入和加速的結構中,已經證明了在電子能量擴散、穩定性方面能夠有效優化。為達到自由電子雷射所要求的電子能量擴散低於0.1\%,本論文的重點是控制注入以及優化加速電子的品質,將比較兩種不同的方法。第一部分是利用稱為衝擊波前沿的超音速現象來刺激瞬間注入。當超音速氣流受到尖銳邊緣的干擾時,會產生衝擊波前緣,並在傳播的雷射脈沖和衝擊波前緣區域的交叉處刺激注入。可以通過調整鋒利邊緣的位置來調整激波前端,從而調整激波前端的位置和角度。先藉由調整雷射電漿參數將加速電子優化至單能,並討論各項參數如何有效的降低加速電子束之能量擴散。第二部分,延續此研究我們發現特定條件下的注入方式具有的不同於多數研究的衝擊波注入方式,我們將深入分析這項注入機制的特色。為了比較這兩種機制,我們將重現普遍研究常見的衝擊波注入,並且透過軌跡追蹤比較兩種注入的加速電子性質差異。此外,利用這樣的注入機制配合傾斜角度的衝擊波能產生明顯的不對稱注入,這有可能使產生的電子加速器輻射具有極化的性質。在 500MeV 的峰值能量,電子束的能量擴散小於 2 \% 。

\end{abstract*}
