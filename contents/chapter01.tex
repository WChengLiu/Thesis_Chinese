% !TeX root = ../main.tex

\chapter{緒論}





\section{雷射電漿加速器}
自從 Tajima 和 Dawson 於 1979 年首次理論預測基於等離子體的電子加速以來,已經發表了大量的實驗架設和改進的模型。當前的模型預測在實際實驗條件下能產生約 10GeV 電子束,勞倫斯伯克利國家實驗室 (LBNL) 的一個團隊在 4.2 .GeV突破當前實驗實現的峰值電子能量記錄僅使用 9 厘米。
因此,這些機器目前正在達到與經典射頻 (RF cavity) 加速器相當的電子能量,並且顯著地減少佔用空間。作為比較,斯坦福線性加速器中心 (SLAC) 的線性相干光源 (LCLS) 使用 1 公里長的加速器來實現 10 GeV 峰值能量的電子束。
然而RF加速器在其他電子束品質方面勝過雷射電漿的電子源,這些參數決定了加速電子束在自由電子雷射 (FEL) 或碰撞實驗中的可用性。其中之一是電子束參數的穩定性。
1994 年在英國摩德納等人的盧瑟福阿普爾頓實驗室。在加州大學洛杉磯分校 (UCLA) 的雷射拍頻狀態中顯示加速尾場存在近 10 年後,在自調變雷射電漿尾場狀態中顯示了峰值能量為 44MeV 的電子。不同之處在於,電子第一次不是從射頻源外部注入,而是在通常被稱為破波(wavebreaking)注入的過程中從背景電漿中捕捉。
隨著外部注入的必要性減少,雷射電漿加速實驗變得更加容易。在過去的二十年中,雷射電漿加速實驗幾乎完全採用從背景電漿中捕獲的加速電子的方式,各式不同的注入方法也成為目前主流研究方向。
為了從實驗上研究雷射尾場加速(LWFA)機制,研究人員面臨了一些挑戰。如果想研究 LWFA 裝置中注入電子束的演化,電漿分佈需要盡量保持不變也就是固定初始條件,而在通常為幾毫米的小規模加速器實驗中,改變加速部分的參數同時製造恆定背景電子的局部注入是非常困難的。
從力學角度來看,無論注入條件如何,與相比於尾場加速中獲得的動量,最初電子在雷射傳播方向上獲得的動量非常小。因此加速度長度和電漿密度,適用於研究縱向電子動量增益,目前已有多個研究開發出相關模型,並且與實驗結果有很好的一致性。
\section{自由電子雷射對電子源的限制}
自由電子雷射是於1970年代首次引入的概念,是一種從相對論性電子束中提取能量轉換為雷射的裝置。
其產生輻射光譜範圍非常廣,甚至能產生傳統雷射光源無法到達的x ray、EUV波長。
這項優勢是使自由電子雷射成為對於材料、化學、生物等研究有著極大幫助的重要工具。
其光源產生原理是將相對論性的自由電子送進稱為擺動器(wiggler )或是聚頻磁鐵(undulator)的週期性磁場。
週期性磁場會讓電子產生橫向震盪,這導致電子會在傳播方向上產生電磁輻射,當聚頻磁鐵與此輻射可以結合產生波拍即能產生脈衝光。
從物理角度而言,最關鍵的環節就是將電子匯聚成多個小於光波波長的小型電子束,使得這項技術對於電子束的品質參數的要求相當高,如能量分散(energy spread)須小於0.1\%\ref{PhysRevX.2.031019},這同時也是目前雷射電漿加速正在面臨的挑戰。
